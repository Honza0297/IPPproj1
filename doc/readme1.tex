\documentclass[a4paper, 11pt]{article}

\usepackage[czech]{babel}
\usepackage{times}
\usepackage[text={17cm,24cm}, top=3cm, left=2cm]{geometry}
\usepackage[utf8]{inputenc}
\setlength{\headheight}{20.0mm}
\usepackage{fancyhdr}
\pagestyle{fancy}
\usepackage{graphics}
\usepackage {array}
\usepackage{pdflscape}
\usepackage[czech, ruled, vlined, linesnumbered, longend, noline]{algorithm2e}
\usepackage{multirow}

\begin{document}
\catcode`\-=12 %Mělo by to vyřešit problém s cline... jestli ne, tak už nevím
%Uvodni strana
\pagestyle{fancy}
\fancyhf{}
\fancyhead[R]{\scalebox{0.5}{\includegraphics{logo.png}}}
\fancyhead[L]{\Large Jan Beran\\ \texttt{xberan43}\\ \today}

\setlength{\parindent}{0mm}
\large{\textbf{Implementační dokumentace k 1. úloze do IPP 2018/2019}}\\\
\large{\textbf{Jméno a příjmení: Jan Beran}}\\\
\large{\textbf{Login: xberan43}}\\\

\large{\textbf{Úvod}}\\\
Tato dokumentace stručně popisuje skript parse.php, který je součástí odevzdání.\\

\large{{\textbf{Zadání}}\\
Zadáním bylo vytvořit skript typu filtr, který převede vstupný kód v jazyce IPPcode19 zadávaný na standardní vstup do jeho XML reprezentace, kterou následně vytiskne na standardní výstup. Přitom zkontroluje jeho lexikální a syntaktickou správnost.\\

\large{{\textbf{Implementace}}\\
Skript je implementován funkcionálně, bez použití objektového návrhu.\\
Vzhledem k jednoduchosti jazyka IPPcode19 zde nebyla použita žádná gramatika ani konečněstavové řízení, místo něj je zde použita kombinace regulárních výrazů a klasické rozhodovací logiky. \\
Skript sám přijímá jediný parametr, -{}-help, po jehož zadání se vypíše krátká nápověda a popis programu.\\
Pro generování XML je využito SimpleXML a výsledný XML kód je pro větši přehlednost tisknut se zalomením konce řádku po každé XML značce. Tato XML reprezentace je generována v průběhu analýzy, ovšem k samotnému zobrazení na standardním výstupu dochází až poté, co je dokončená analýza. \\
Chybové stavy jsou ošetřeny standardně dle zadání, tedy ukončením skriptu a vrácením příslušného návratového kódu.\\

\large{{\textbf{Spouštění aplikace}}\\
Aplikace se spouští buď bezparametricky nebo s parametrem -{}-help, který vypíše krátkou nápovědu. Pokud je skript spuštěn bez něj, poté na standardním vstupu očekává kód v jazyce IPPcode19 včetně jeho hlavičky. K výpisu XML kódu na standardní výstup dojde až ve chvíli, kdy je dokončeno zadávání vstupního kódu.\\ 


\large{{\textbf{Závěr}}\\
Skript byl implementován samostatně a odpovídá všem náležitostem zadání.\\
\end{document}





































